\documentclass[12pt]{mwrep}
\usepackage{polski}
\usepackage[utf8]{inputenc}
\usepackage[T1]{fontenc}
\usepackage{times}



\usepackage[margin=20mm, left=30mm]{geometry}

\usepackage{newtxtext}
\usepackage{newtxmath}
\usepackage{amsmath}
\usepackage{bm}
\usepackage{mathtools}
\mathtoolsset{showonlyrefs}


\usepackage{tabularx}
\usepackage{array}
\newcolumntype{Y}{>{\centering\arraybackslash}X}
\newcolumntype{Z}{>{\centering\arraybackslash}p}
\usepackage{multirow}
\usepackage[table]{xcolor}
\usepackage{array}
\setlength\arrayrulewidth{1pt}
\newcommand{\x}{\overline{d}}
\usepackage{hyperref}


\renewcommand{\thesection}{{\hspace{-10mm}}\arabic{section}}
\renewcommand{\thesubsection}{{\hspace{-5mm}}\arabic{section}.\arabic{subsection}}

\usepackage{enumitem}
\usepackage{float}
\usepackage{longtable}
\let\alpha\alphaup
\usepackage{graphicx}
\usepackage{rotating}
\usepackage{subcaption}

\newcommand{\code}[1]{\texttt{#1}}

\newcommand{\dd}{\text{d}}

%%%%%%%%%%%%%%%%%%%%%%%%%%%%%% Malec - preambuła







%%%%%%%%%%%%%%%%%%%%%%%%%%%%%% Budnik - preambuła
\newcommand{\indep}{\perp \!\!\! \perp}
%\usepackage{titlesec}
%\titleclass{\subsubsection}{straight}[\subsection]
%\newcounter{\subsubsection}[subsubsection]
%\renewcommand{\thesubsubsection}{{\hspace{-2mm}}\arabic{section}.\arabic{subsection}.\arabic{subsubsection}}











\begin{document}%%%%%%%%%%%%%%%%%%%%%%%%%%%%
	\begin{center}
		{\Large\textbf{Symulacje Komputerowe}}
	\end{center}
	\begin{center}
		Raport: \textbf{1}
	\end{center}
	
	\noindent Temat sprawozdania \dotfill \textbf{Coś kreatywnego} \dotfill\dotfill\\
	Nazwisko i Imię prowadzącego kurs \dotfill \textbf{dr Michał Balcerek} \dotfill\dotfill	\newline\newline
	
	
	\noindent\begin{tabularx}{\textwidth}{|X |X|}
		\hline
		Wykonawca: & \\\hline
		\begin{center}
			Imię i Nazwisko,\\ nr indeksu
		\end{center} &  \begin{center}
			Kacper Budnik, 262286\\
			Szymon Malec, 262276
		\end{center}\\\hline
		Wydział & Wydział matematyki, W13 \\\hline
		Termin zajęć: & Wtorek,\vphantom{ $11^{1^{5}}$} $15^{15}$\\\hline
		Numer grupy ćwiczeniowej & T00-70d \\\hline
		Data oddanie sprawozdania: & \today \\\hline
		\textbf{Ocena końcowa} &\\\hline
		
	\end{tabularx}\newline\newline
	
	
	\noindent\textbf{Adnotacje dotyczące wymaganych poprawek oraz daty otrzymania poprawionego sprawozdania}
	
	
	
	\newpage

	\section{Wstęp - Koniec }
	



	\section{Liniowy generator kongruentny - Kiedyś}


%%%%%%%%%%%%%%%%%%%%%%%%%%%%%%%%%%%%%%%%1
	
	\section{Metoda odwrotnej dystrybuanty - Malec}
	\subsection{Teoria}
	\subsection{Algorytm}
	\subsection{Przykład}


%%%%%%%%%%%%%%%%%%%%%%%%%%%%%%%%%%%%%%%%2

	
	\section{Metoda akceptacji i odrzucenia\textsuperscript{\cite{A-O}} - Budnik}
	\subsection{Opis}
	Metoda akceptacji i odrzucenia służy do generowania zmiennej losowej \textbf{X} przy użyciu innych zmiennych. By móc wykorzystać tą metodę muszą być spełnione:
	\begin{itemize}[leftmargin=10mm]
		\item Potrafimy efektywnie generować inną zmienną losową \textbf{Y}
		\item Zmienne \textbf{X} oraz \textbf{Y} muszą być skupione na tym samym zbiorze
		\item Potrafimy wyznaczyć stałą $c$ taką że $\dfrac{\mathbb{P}(X=i)}{\mathbb{P}(Y=i)}\leqslant c$ dla każdego $i$
	\end{itemize}
	Jeśli są spełnione powyższe założenia możemy użyć poniższego algorytmu do generowania zmiennej \textbf{X}.
	\subsubsection{Algorytm}
	\begin{enumerate}[leftmargin=10mm]
		\item Generuj jedną realizację \textbf{Y}
		\item Generuj U\textasciitilde U(0,1), $\textbf{U}\boldsymbol{\indep} \textbf{Y}$
		\item Jeśli $\textbf{U}\leqslant\frac{p_\textbf{Y}}{cq_\textbf{Y}}$ zwróć \textbf{X}=\textbf{Y}, w przeciwnym wróć do 1.
	\end{enumerate}
	Prawdopodobieństwo że zmienna zostanie zaakceptowana wynosi
	$$\mathbb{P}(\text{'wartość zaakceptowana'})=\frac{1}{c}$$
	zatem by algorytm był wydajny stała $c$ powinna być jak najmniejsza. Średnia liczba powtórzeń algorytmu wynosi $c$.
	\textbf{To było dyskretne, jeszcze potrzebne ciągłe}
	
	\subsection{Przykład}


%%%%%%%%%%%%%%%%%%%%%%%%%%%%%%%%%%%%%%%%3
	
	\section{Metoda splotowa - Malec}
	
	\section{Metoda kompozycji - Malec}




%%%%%%%%%%%%%%%%%%%%%%%%%%%%%%%%%%%%%%%%4
	
	\section{Metoda Boxa-Mullera - Budnik}
	
	\section{Metoda biegunowa - Budink}



	
	\section{Zakończenie - Początek}
	
	
	\begin{thebibliography}{1}
		\bibitem{A-O}
		\url{https://youtu.be/NFmbgbyj5M0?t=1323}

	\end{thebibliography}
	
\end{document}