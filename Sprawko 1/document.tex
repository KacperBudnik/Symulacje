\documentclass[12pt]{mwrep}
\usepackage{polski}
\usepackage[utf8]{inputenc}
\usepackage[T1]{fontenc}
\usepackage{times}



\usepackage[margin=20mm, left=30mm]{geometry}

\usepackage{newtxtext}
\usepackage{newtxmath}
\usepackage{amsmath}
\usepackage{bm}
\usepackage{mathtools}
\mathtoolsset{showonlyrefs}


\usepackage{tabularx}
\usepackage{array}
\newcolumntype{Y}{>{\centering\arraybackslash}X}
\newcolumntype{Z}{>{\centering\arraybackslash}p}
\usepackage{multirow}
\usepackage[table]{xcolor}
\usepackage{array}
\setlength\arrayrulewidth{1pt}
\newcommand{\x}{\overline{d}}
\usepackage{hyperref}


\renewcommand{\thesection}{{\hspace{-10mm}}\arabic{section}}
\renewcommand{\thesubsection}{{\hspace{-5mm}}\arabic{section}.\arabic{subsection}}
%\usepackage{titlesec}
%\titlespacing*{\section}{0pt}{100pc}{3pc}
\usepackage{enumitem}
\usepackage{float}
\usepackage{longtable}
\let\alpha\alphaup
\usepackage{graphicx}
\usepackage{rotating}
\usepackage{subcaption}

\newcommand{\code}[1]{\texttt{#1}}

\newcommand{\dd}{\text{d}}




\begin{document}
	\begin{center}
		{\Large\textbf{Analiza sygnałów}}
	\end{center}
	\begin{center}
		Sprawozdanie: \textbf{1}
	\end{center}
	
	\noindent Temat sprawozdania \dotfill \textbf{Wyznaczenie pojemności kondensatora} \dotfill\dotfill\\
	Nazwisko i Imię prowadzącego kurs \dotfill \textbf{dr inż. Ireneusz Augustyniak} \dotfill\dotfill	\newline\newline
	
	
	\noindent\begin{tabularx}{\textwidth}{|X |X|}
		\hline
		Wykonawca: & \\\hline
		\begin{center}
			Imię i Nazwisko,\\ nr indeksu
		\end{center} &  \begin{center}
			Kacper Budnik, 262286\\
			Tomasz Hałas, 254637
		\end{center}\\\hline
		Grupa & 9 \\\hline
		Wydział & Wydział matematyki, W13 \\\hline
		Termin zajęć: & Środa,\vphantom{ $15^{1^{5}}$} $15^{15}$\\\hline
		Numer grupy ćwiczeniowej & T00-68c \\\hline
		Data oddanie sprawozdania: & \today \\\hline
		\textbf{Ocena końcowa} &\\\hline
		
	\end{tabularx}\newline\newline
	
	
	\noindent\textbf{Adnotacje dotyczące wymaganych poprawek oraz daty otrzymania poprawionego sprawozdania}
	
	
	
	\newpage %%%%%%%%%%%%%%%%%%%%%%%%%%%%%%%%%%%%%%%%%% Hałas
	
	\section{Wstęp}
	
	\section{Liniowy generator kongruentny}
	
	\section{Metoda odwrotnej dystrybuanty}
	\subsection{Teoria}
	\subsection{Algorytm}
	\subsection{Przykład}
	
	\section{Metoda akceptacji i odrzucenia}
	
	\section{Metoda splotowa}
	
	\section{Metoda kompozycji}
	
	\section{Metoda Boxa-Mullera}
	
	\section{Metoda biegunowa}
	
	\section{Zakończenie}
	
	
	\begin{thebibliography}{1}%%%%%%%%%%%%%%%%%%%%%%%%%%%%%%%%%%%%%%%%%% Na koniec link do strony Augustyniaka itp (na początku Budnik)
		\bibitem{RC}
		\url{http://prac.im.pwr.wroc.pl/~augustyniak/}
		\bibitem{code}
		\url{http://prac.im.pwr.wroc.pl/~augustyniak/dydaktyka.html}
		%https://www.uj.edu.pl/c/document\_library/get\_file?uuid=582ee3f9-39b6-48fe-aaf0-681008fc8b18\&groupId=5046939 \\
		%https://en.wikipedia.org/wiki/RC\_circuit\\
	\end{thebibliography}
	
\end{document}