\documentclass[12pt]{mwrep}
\usepackage{polski}
\usepackage[utf8]{inputenc}
\usepackage[T1]{fontenc}
\usepackage{times}



\usepackage[margin=20mm, left=30mm]{geometry}

\usepackage{newtxtext}
\usepackage{newtxmath}
\usepackage{amsmath}
\usepackage{bm}
\usepackage{mathtools}
\mathtoolsset{showonlyrefs}


\usepackage{tabularx}
\usepackage{array}
\newcolumntype{Y}{>{\centering\arraybackslash}X}
\newcolumntype{Z}{>{\centering\arraybackslash}p}
\usepackage{multirow}
\usepackage[table]{xcolor}
\usepackage{array}
\setlength\arrayrulewidth{1pt}
\newcommand{\x}{\overline{d}}
\usepackage{hyperref}


\renewcommand{\thesection}{{\hspace{-10mm}}\arabic{section}}
\renewcommand{\thesubsection}{{\hspace{-5mm}}\arabic{section}.\arabic{subsection}}

\usepackage{enumitem}
\usepackage{float}
\usepackage{longtable}
\let\alpha\alphaup
\usepackage{graphicx}
\usepackage{rotating}
\usepackage{subcaption}

\newcommand{\code}[1]{\texttt{#1}}

\newcommand{\dd}{\text{d}}

%%%%%%%%%%%%%%%%%%%%%%%%%%%%%% Malec - preambuła







%%%%%%%%%%%%%%%%%%%%%%%%%%%%%% Budnik - preambuła











\begin{document}%%%%%%%%%%%%%%%%%%%%%%%%%%%%
	\begin{center}
		{\Large\textbf{Symulacje Komputerowe}}
	\end{center}
	\begin{center}
		Raport: \textbf{1}
	\end{center}
	
	\noindent Temat sprawozdania \dotfill \textbf{Coś kreatywnego} \dotfill\dotfill\\
	Nazwisko i Imię prowadzącego kurs \dotfill \textbf{dr Michał Balcerek} \dotfill\dotfill	\newline\newline
	
	
	\noindent\begin{tabularx}{\textwidth}{|X |X|}
		\hline
		Wykonawca: & \\\hline
		\begin{center}
			Imię i Nazwisko,\\ nr indeksu
		\end{center} &  \begin{center}
			Kacper Budnik, 262286\\
			Szymon Malec, 262276
		\end{center}\\\hline
		Wydział & Wydział matematyki, W13 \\\hline
		Termin zajęć: & Wtorek,\vphantom{ $11^{1^{5}}$} $15^{15}$\\\hline
		Numer grupy ćwiczeniowej & T00-70d \\\hline
		Data oddanie sprawozdania: & \today \\\hline
		\textbf{Ocena końcowa} &\\\hline
		
	\end{tabularx}\newline\newline
	
	
	\noindent\textbf{Adnotacje dotyczące wymaganych poprawek oraz daty otrzymania poprawionego sprawozdania}
	
	
	
	\newpage

	\section{Wstęp - Koniec }
	



	\section{Liniowy generator kongruentny - Kiedyś}


%%%%%%%%%%%%%%%%%%%%%%%%%%%%%%%%%%%%%%%%1
	
	\section{Metoda odwrotnej dystrybuanty - Malec}
	\subsection{Teoria}
	\subsection{Algorytm}
	\subsection{Przykład}


%%%%%%%%%%%%%%%%%%%%%%%%%%%%%%%%%%%%%%%%2

	
	\section{Metoda akceptacji i odrzucenia\textsuperscript{\cite{A-O}} - Budnik}



%%%%%%%%%%%%%%%%%%%%%%%%%%%%%%%%%%%%%%%%3
	
	\section{Metoda splotowa - Malec}
	
	\section{Metoda kompozycji - Malec}




%%%%%%%%%%%%%%%%%%%%%%%%%%%%%%%%%%%%%%%%4
	
	\section{Metoda Boxa-Mullera - Budnik}
	
	\section{Metoda biegunowa - Budink}



	
	\section{Zakończenie - Początek}
	
	
	\begin{thebibliography}{1}
		\bibitem{A-O}
		\url{https://youtu.be/NFmbgbyj5M0?t=1323}

	\end{thebibliography}
	
\end{document}