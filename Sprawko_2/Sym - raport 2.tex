\documentclass[12pt]{mwart}

\usepackage{polski}
\usepackage[utf8]{inputenc}
\usepackage{mathtools,amsthm,amssymb,icomma,upgreek,xfrac,graphics,scrextend,float,tabularx,hyperref,multicol,array,caption,enumitem}
\usepackage[table,xcdraw]{xcolor}

\mathtoolsset{mathic}
\raggedbottom
\graphicspath{ {./images/} }
\renewcommand{\refname}{Źródła}
\captionsetup{justification=raggedright,singlelinecheck=true}

\newcommand{\indep}{\perp \!\!\! \perp}
\begin{document}
	
	\begin{center}
		{\Large\textbf{Symulacje komputerowe}}
	\end{center}
	\begin{center}
		Raport 2
	\end{center}
	
	\noindent Temat: \ \textbf{Proces Ryzyka i ruch Browna}\\
	Imię i Nazwisko prowadzącego kurs: \ \textbf{Dr Michał Balcerek}	\newline\newline
	
	
	\noindent\begin{tabularx}{\textwidth}{|X |X|}
		\hline
		\begin{center}
			Imię i Nazwisko,\\ nr indeksu
		\end{center} &  \begin{center}
			Kacper Brudnik, 262286\\
			Szymon Malec, 262276
		\end{center}\\\hline
		Wydział: & Wydział matematyki, W13 \\\hline
		Dzień i godzina zajęć: & Wtorek,\vphantom{ $11^{1^{5}}$} $11^{15}$\\\hline
		Kod grupy ćwiczeniowej: & T00-70d \\\hline
		Data oddania raportu: & 26.06.2022 \\\hline
		\textbf{Ocena końcowa} &\\\hline
	\end{tabularx}\newline\newline
	
	\noindent\textbf{Adnotacje i uwagi:}
	
	\newpage
	
	
	\section{Wstęp}
	\noindent 
	
	
	
	\section{Zadanie 1}
	
	\subsection{Jednorodny proces Poissona}
	\noindent Jednorodnym procesem Poissona z intensywnością $\lambda > 0$ nazywamy proces liczący $\{N_t\}_{t \geq 0}$, który spełnia:
	\begin{itemize}
		\item $N_0 = 0$,
		\item $N_t$ ma niezależne przyrosty,
		\item $N_t$ ma stacjonarne przyrosty,
		\item $N_t \sim \mathcal{P}oiss(\lambda t)$.
	\end{itemize}
	
	\noindent \textbf{Algorytm}
	\begin{enumerate}
		\item Definiuj $I = 0$, $t=0$.
		\item Generuj $X \sim \mathcal{E}xp(\lambda)$.
		\item $t = t + X$. Jeśli $t > T$, przerwij i zwróć $\{S_i\}$. W przeciwnym razie $I = I + 1$, $S_I = t$.
		\item Wróć do kroku 2.
	\end{enumerate}
	Otrzymane z powyższego algorytmu $\{S_i\}$ to czasy oczekiwania na i-ty skok od startu trajektorii.
	
	
	\subsection{Klasyczny proces Ryzyka}
	\noindent Klasycznym procesem Ryzyka nazywamy proces stochastyczny opisujący kapitał firmy ubezpieczeniowej postaci
	$$ R_t = u + ct - \sum_{i=1}^{N_t} X_i, $$
	gdzie
	\begin{itemize}[label=\textbf{.}]
		\item $u > 0$ -- kapitał początkowy,
		\item $ct$ -- stałe przychody ($c > 0$),
		\item $N_t$ -- jednorodny proces Poissona z intensywnością $\lambda > 0$,
		\item $X_i$ -- ciąg niezależnych zmiennych losowych o tym samym rozkładzie opisujący straty, $\mathrm{E}X_i = \mu$, $X_i > 0$.
	\end{itemize}

	\noindent \textbf{Algorytm}
	\begin{enumerate}
		\item Generuj $N_t$ na $[0, T]$.
		\item Generuj $X_1, \dots, X_{N_t}$.
		\item Zwróć $R_t = u + ct - \sum_{i=1}^{N_t} X_i$.
	\end{enumerate}
	
	
	\subsection{Dopasowanie modelu do danych}
	
	\subsection{Prawdopodobieństwo ruiny}
	
	
	
	
	\section{Zadanie 2}
	\subsection{Proces Wienera}
	\noindent Procesem Wienera (Ruchem Browna) nazywamy proces $\{W_t\}_{t\geqslant0}$, który spełnia:
	\begin{itemize}[leftmargin=10mm]%, label=$\boldsymbol{\cdot}$]
		\item $W_0=0$,
		\item $W_t$ ma niezależne przyrosty,
		\item $W_t$ ma stacjonarne przyrosty,
		\item $W_t \sim \mathcal{N}(0, t)$,
		\item $W_t$ ma ciągłe trajektorie.
	\end{itemize}	
	\noindent \textbf{Algorytm}\\
	\noindent Naszym celem jest wygenerowanie wektora $\left[W_{t_0}, W_{t_1}, \dots, W_{t_n}\right]$, gdzie
	\begin{equation}
		t_i=ih, \quad i=0,1,\dots,n, \quad h=T/n
	\end{equation}
	\begin{equation}
		t_i=ih, \quad i\in\left[n\right]=\left\{0, 1, \dots, n\right\}, \quad h=T/n
	\end{equation}
	\begin{equation}
		t_i=ih, \quad i\in\mathbb{N}_0^{\leqslant n}, \quad h=T/n.
	\end{equation}
	W tym celu stosujemy poniższy algorytm
	\begin{enumerate}[leftmargin=10mm]
		\item Generuj realizację zmiennych $\xi_i\text{ iid }\mathcal{N}(0,1), \,i\in\left[n-1\right]$,
		\item $W_{t_0}=W\left(0\right)=0$,
		\item $W_{t_{i+1}} = W_{t_{i}}+\sqrt{h}\xi_i,\quad i\in\left[n-1\right]$%\in\mathbb{N}^{n}$.
	\end{enumerate}

	\subsection{Średni czas wyjścia procesu Wienera}
	\noindent Niech $\left\{B^x_t\right\}_{t\geqslant0}$ będzie ruchem Browna startującym z $x\in\mathbb{R}$, a $\tau^x$ czasem wyjścia tego procesu z ustalonego przedziału $[a, b]$, czyli
	\begin{equation}
		\tau^x=\inf\left\{t\geqslant0:B^x_t\notin[a, b]\right\}
	\end{equation}
	\noindent dla $x$ z przedziału $[a,b]$. Łatwo można pokazać, że $B^{x+y}_t\in[a ,b] \iff B^x_t\in[a-y, b-y]$.
%	\\Ponieważ $W_t+x\in[a, b] \iff W_t\in[a-x, b-x]$ \\
%	$B^x_t\in[a,b]\iff B^y_t\in[a-x+y, b-x+y]$\\widzimy, że czas wyjścia nie zależy bezpośrednio od wyboru punktów $a$ oraz $b$, a od punktu początkowego oraz długości przedziału.
	Dodatkowo, ponieważ ruch Browna jest $\frac{1}{2}$-samopodobny($W_t\overset{d}{=}\sqrt{\Delta}W_{t/\Delta}$), zachodzi
	\begin{equation}
		B^x_t=W_t+x\overset{d}{=}\Delta W_{t/\Delta^2}+x = \Delta\left(W_{t/\Delta-\frac{x}{\Delta}}\right)=B^{x/\Delta}_{t/\Delta^2}
	\end{equation}
	Korzystając z tych dwóch własności pokażemy, że
	\begin{equation}
		\begin{split}
			\tau^x&\overset{d}{=}\inf\left\{\Delta^2\frac{t}{\Delta^2}\geqslant0:\Delta B^{x/\Delta}_{t/\Delta^2}\notin[a, b]\right\} =\Delta^2\inf\left\{t^*\geqslant0:B^{x/\Delta}_{t^*}\notin\left[\frac{a}{\Delta},\frac{b}{\Delta}\right]\right\}\\
			&=\Delta^2\inf\left\{t^*\geqslant0:B^{(x-y)\Delta}_{t^*}\notin\left[\frac{a-y}{\Delta},\frac{b-y}{\Delta}\right]\right\}
		\end{split}
	\end{equation}
	Wybierając teraz $\Delta=(b-a)$ oraz $y=a$ otrzymujemy
	\begin{equation}\label{eq:tau_przeskalowane}
		\tau^x\overset{d}{=}(b-a)^2\inf\left\{t\geqslant0:B^z_t\in[0,1]\right\},
	\end{equation}
	gdzie $z$ jest "przeskalowanym" $x$ na odcinek $[0, 1]$.\\
	Dzięki tym przekształceniom widzimy, że $\mathbb{E}\tau^x$ zwiększa się z kwadratem długości przedziału jaki rozważamy oraz zależy ona od odległości $x$ od jednego końca przedziału względem drugiego. Z tego powodu później będziemy rozważać jedynie czas wyjścia z przedziału $[0, 1]$.
	
	
	
	
%	Naszym celem jest oszacowanie średniego czasu wyjścia procesu Wienera startującego z $x\in\mathbb{R}$, w zależności od wyboru $x$.

















	\newpage
	
	\section{Podsumowanie}
	
	
	
	\newpage
	\begin{thebibliography}{1}
		\bibitem{dane}
		\url{}
	\end{thebibliography}


\end{document}